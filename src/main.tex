\documentclass[a4paper, 11pt, german ]{article}
\usepackage{babel}
%\usepackage{biblatex}
\usepackage{titling} % ja es gibt andere wege wie man das in ohne dieses package machen kann aber habe keinen zum funktioniern bekommen
\usepackage{geometry}
\usepackage{setspace}

\usepackage{hyperref}
\usepackage{digsig} % sollte in TEX live installiert sein wird nur für signaturen benötigt


\usepackage[
backend=biber,
style=apa
]{biblatex}
\addbibresource{./src/quellen.bib}


% der Geometrie
\geometry{top=2.5cm, bottom=2.0cm , left=4.0cm , right=2.0cm ,a4paper}

% definiert true and false
\def\True{1}
\def\False{0}


% alles was irgendwie mit fonts, font groessen und so zu tun hat 
\usepackage{helvet}
\renewcommand{\familydef\usepackage{digsig}ault}{\sfdefault}

\onehalfspacing % sollte denn Zeilenabstand auf 1.5 setzten 


\def\quellen{Quellenverzeichnis} % Quellen

% setting date to the korrect format 
\usepackage[ddmmyyyy]{datetime}
\renewcommand{\dateseparator}{.}


\begin{document}

% ALl of your information should be placed here

% 
\author{Name} %z.b. Max Muserman
\def\MatrNummer{00000} % example for a matrikelnumber


% usually unchanged things 
\date{today}





% Hauptseiten Zeug
\begin{titlepage}
\if\BachlorThesis\True
	\newgeometry{top=1.31cm, bottom=0.49cm , left=2.29cm , right=2.29cm ,a4paper}
	\vspace*{20ex}
	\begin{center}
		
		\textbf{\thetitle}\\
		\vspace{10ex}
		Bachelor-Thesis\\
		an der \\
		\Hochschulname\\
		im Rahmen des Studiengangs \\
		\Studiengang \\
		zur/ zum \Abschlussbezeinung (B.A.) bzw. (B.Sc.) \\
	
		\vspace{1ex}
		vorgelegt von \\
		\theauthor \\
		aus\\
		\Addresse\\
		\Email\\
		
		\vspace{1ex}
		Matr.-Nr.: \MatrNummer\\
		
		\vspace{10ex}
		Erstgutachter: \Pruefer \\
		Zweitgutachter: \ZweitGutachter \\
		\vspace{1ex}
		Abgabe : \thedate
	\end{center}

\else


	\newgeometry{top=1.31cm, bottom=0.49cm , left=2.29cm , right=2.29cm ,a4paper}
\raggedright
\vspace*{20ex}
\Hochschulname \\
\Studiengang\\
Betreut durch: \Pruefer

\vspace{10ex}
\begin{center}
	\ArtArbeit\\
	\vspace{1ex}
	im Fach \Modul im \Semester\\
	
	\vspace{10ex}
	\textbf{\thetitle}

\end{center}

\vspace{10ex}
\theauthor\\
\Addresse\\
Matr.-Nr.: \MatrNummer\\
\vspace{1ex}
E-Mail-Adresse \Email\\
\vspace{3ex}
Abgabedatum: \thedate

\fi
\restoregeometry
\end{titlepage}

% Table of Contents und Sperrvermerk
\newpage
\pagenumbering{Roman}
\tableofcontents




% nSperrvermerk wennötig
% einstellbar in Information.tex
\if\Sperrvermerk \True
\newpage
 \section*{Sperrvermerk}
 Die in dieser schriftlichen Arbeit enthaltenen Informationen sind vertraulichund ausschließlich
 für die entsprechenden Prüfer und Mitarbeiter der \Hochschulname bestimmt.
Jeglicher Zugriff durch andere Personen, Veröffentlichung, Vervielfältigung,
Verteilung oder sonstige in diesem Zusammenhang stehende Handlung sind nur mit ausdrücklicher
Genehmigung der Verf. und \Firma\space zulässig. Darüber hinaus besitzt der
Verf. die urheberrecht\-lichen Ansprüche und \Firma\space die Nutzungsrechte
an bzw. aus dieser Arbeit.
\newpage
\fi


% including most of the random text stuff
\pagenumbering{arabic}


\if\Makefile\True
   Lorem ipsum dolor sit amet, consectetur adipiscing elit. Phasellus porttitor, est et facilisis cursus, augue ipsum faucibus ipsum, nec elementum purus tellus non massa. Nulla eu eros ex. Lorem ipsum dolor sit amet, consectetur adipiscing elit. Aenean a laoreet quam. Interdum et malesuada fames ac ante ipsum primis in faucibus. Aenean gravida posuere nisi. In hac habitasse platea dictumst. Cras nibh felis, feugiat eu pulvinar a, iaculis at eros. Sed at mauris nec mauris hendrerit fermentum et id nulla. Sed cursus viverra arcu vel maximus. Quisque laoreet volutpat lacus, sit amet malesuada tellus viverra eget. Donec pharetra dictum vehicula. \cite{interpretation-of-Computer-Programs}

  \include{./src/Main_part.tex}
  \include{./src/Ending.tex}
\else
   Lorem ipsum dolor sit amet, consectetur adipiscing elit. Phasellus porttitor, est et facilisis cursus, augue ipsum faucibus ipsum, nec elementum purus tellus non massa. Nulla eu eros ex. Lorem ipsum dolor sit amet, consectetur adipiscing elit. Aenean a laoreet quam. Interdum et malesuada fames ac ante ipsum primis in faucibus. Aenean gravida posuere nisi. In hac habitasse platea dictumst. Cras nibh felis, feugiat eu pulvinar a, iaculis at eros. Sed at mauris nec mauris hendrerit fermentum et id nulla. Sed cursus viverra arcu vel maximus. Quisque laoreet volutpat lacus, sit amet malesuada tellus viverra eget. Donec pharetra dictum vehicula. \cite{interpretation-of-Computer-Programs}

  \include{./Main_part.tex}
  \include{./Ending.tex}
  
\fi

\printbibliography

% Ehrenwörtliche Erklärung

\newpage
\section*{Ehrenwörtliche Erklärung}
Hiermit versichere ich, dass die vorliegende Arbeit von mir selbstständig und ohne unerlaubte Hilfe
angefertigt worden ist, insbesondere, dass alle Stellen, die wörtlich oder sinngemäß aus
Veröffentlichungen entnommen sind, durch Zitate als solche kenntlich gemacht wurden.
Diese Versicherung bezieht sich auch auf die in der Arbeit verwendete bildliche Darstellungen,
Tabellen, Zeichnungen, Skizzen, graphischen Darstellungen und dergleichen sowie auch für die
Verwendung von text- oder codegenerierenden KI-Werkzeugen als Quelle.
\vspace{1ex}\\
Die Arbeit hat in gleicher oder ähnlicher Form noch keiner Prüfungsbehörde vorgelegen und ist nicht
veröffentlicht. Sie wurde nicht, auch nicht auszugsweise, für eine andere Prüfungs- oder
Studienleistung verwendet.
\vspace{1ex}\\
Ich bin damit einverstanden, dass die Arbeit einer elektronischen Plagiatsprüfung unterzogen werden
kann. Die Regelungen der Prüfungsordnung zur Täuschung habe ich zur Kenntnis genommen.

% TODO toggel hinzufügen 
\begin{Form}
  \digsigfield{5cm}{3cm}{Meine Unterschrift}
\end{Form}

\end{document}
